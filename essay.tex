\documentclass[twoside,titlepage,11pt]{article}
\usepackage[a4paper]{geometry}
\usepackage{url,amsmath,amssymb}
\title{The Category of Sets in Higher-Order Logic}
\author{Ramana Kumar\\MPhil in Advanced Computer Science\\Peterhouse}
\begin{document}
\maketitle
\begin{abstract}%{
The foundational difficulties presented by category theory make it difficult to formalise.
Nevertheless, it is possible to formalise category theory in higher-order logic, extended with a type of small sets, in a way that fits naturally into the idiom of the logic but can still express some basic but foundationally-challenging categorical results.
It is not obvious that this approach can be pushed very far.
A more satisfying formalisation of category theory will require a new way to resolve the foundational issues, a different logic, or a new approach to formalisation.
\end{abstract}%}
\section{Introduction}%200%{
Category theory provides a framework for studying various notions common to diverse areas of mathematics.
It has applications in computer science ranging from the semantics of programming languages and formal systems to models of concurrency <refs>.
Since its inception, category theory has challenged the traditional foundations of mathematics, namely in set theory, by dealing with collections that are too large to be sets. 
Several workarounds have been discovered over the years, and it is often possible to set things up so they work fine for the problem at hand, but do not capture the full implied generality of the categorical constructions.

Formalisation is a good way to test the limits of foundational choices, because a formal development must respect those choices throughout.
There can be no resort to arguments or constructions by analogy or implied generalisation in a fully formal development, especially in the kind developed mechanically with a proof assistant.
Several mechanisations of category theory in different frameworks, with varying coverage and purposes, have been developed.
In this essay, we shall investigate the foundational issues and see some of the solutions, including a new refinement of previous work in higher-order logic.

Higher-order logic is a simple but expressive formal system suitable for the formalisation of many pieces of mathematics.
As a result, it is the logic implemented by many proof assistants in which formal mathematics has been developed <refs>.
Higher-order logic has a standard semantics in set theory, and therefore cannot capture large categorical constructions directly.
However, it can be extended with extra axioms asserting the existence of large classes.

We describe a mechanisation of the category of sets, the simplest foundationally tricky category, in higher-order logic so extended.
This development is done in a way that fits naturally into the idiom of the logic but can still express some basic but foundationally-challenging categorical results.
We comment on the limitations of our approach, compare our mechanisation to others, and make brief speculations on what a more satisfying mechanisation of category theory would require and would look like.
%}
\section{Higher-Order Logic}%250%{
%}
\section{Composable Morphisms and Categories}%2000%{
%}
\section{A Category of Classes and of Sets}%1500%{
%}
\section{Binary Products and Other Limits}%250%{
%}
\section{The Yoneda Embedding}%500%{
%}
\section{Conclusions}%500%{
%}
\bibliographystyle{plain}
\bibliography{essay}
\end{document}
% vim:fdm=marker:fmr=%{,%}
