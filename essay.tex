\documentclass[twoside,titlepage,11pt]{article}
\usepackage[a4paper]{geometry}
\usepackage{url,amsmath,amssymb}
\title{The Category of Sets in Higher-Order Logic}
\author{Ramana Kumar\\MPhil in Advanced Computer Science\\Peterhouse}
\begin{document}
\maketitle
\section*{Declaration}%{
\thispagestyle{empty}
I Ramana Kumar of Peterhouse, being a candidate for the MPhil in Advanced Computer Science, hereby declare that this research essay and the work described in it are my own work, unaided except as may be specified below, and that the research essay does not contain material that has already been used to any substantial extent for a comparable purpose.

\vspace{1em}
\noindent Signed

\vspace{1em}
\noindent Date

\vspace{2em}
\noindent Words: 5909 (Maximum: 5000)
%}
\begin{abstract}%{
The foundational difficulties presented by category theory make it difficult to formalise.
Nevertheless, it is possible to formalise category theory in higher-order logic, extended with a type of small sets, in a way that fits naturally into the idiom of the logic but can still express some basic but foundationally-challenging categorical results.
It is not obvious that this approach can be pushed very far.
A more satisfying formalisation of category theory will require a new way to resolve the foundational issues, a different logic, or a new approach to formalisation.
\end{abstract}%}
\section{Introduction}%386/300%{
Category theory provides a framework for studying various notions common to diverse areas of mathematics.
It has applications in computer science ranging from the semantics of programming languages and formal systems (e.g., \cite{CroleCT,JacobsCLTT}) to models of concurrency~\cite{DBLP:conf/csl/CattaniW96}.
Since its inception, category theory has challenged the traditional foundations of mathematics, namely in set theory, by dealing with collections that are too large to be sets. 
Often, it is possible to set things up well for the problem at hand while failing to capture the full generality of the categorical constructions.

Formalisation is a good way to test the limits of foundational choices, because a formal development must respect those choices throughout.
There can be no resort to arguments or constructions by analogy or implied generalisation, especially in formalisations developed mechanically with a proof assistant.
In this essay we present a new mechanisation of category theory\footnote{Our proof scripts are available online at \url{https://github.com/xrchz/HOL/tree/category/examples/category}, as well as in the HOL4 repository (\url{http://sf.net/projects/hol/develop}) from revision <9115?>.}, refining two previous developments in higher-order logic, and investigate the foundational issues.

Higher-order logic (HOL) is a simple but expressive formal system suitable for the formalisation of many pieces of mathematics.
As a result, it is the logic implemented by many proof assistants in which formal mathematics has been developed.
HOL has a standard semantics in set theory, and therefore cannot capture large categorical constructions directly.
However, it can be extended with extra axioms asserting the existence of large classes.

We formalise the category of sets, the simplest foundationally tricky category, in higher-order logic so extended.
We try to fit naturally into the idiom of the logic while retaining the ability to express some basic but foundationally-challenging categorical results.
We comment on the limitations of our approach, compare our mechanisation to others, and make brief speculations on what a more satisfying mechanisation of category theory would require and would look like.

This essay is structured as a high-level tour of our mechanisation, with discussions of foundational issues, and comparisons to existing work, as they come up.
We begin with a brief introduction to HOL, then look at our formalisation of category theory, the category of sets in particular, how we treat the Yoneda embedding, and the start of work on categorical limits.
%}
\section{Higher-Order Logic}%504/500%{
\newcommand{\bool}{\ensuremath{\mathsf{bool}}}
Higher-order logic (HOL) is a formal system that has proved popular for formalising purely mathematical results as well as properties of computer hardware and software systems.
HOL is a natural-deduction calculus with the syntax of the simply typed lambda calculus.
A term of Boolean type is called a \emph{formula}.
HOL has four formulae specified as axioms and a dozen rules of inference; a \emph{theorem}, $\vdash\phi$, is a formula $\phi$ proved from the axioms using the rules.
The HOL logic is implemented by several mechanical proof assistants, including Isabelle~\cite{DBLP:conf/tphol/WenzelPN08}, HOL4~\cite{DBLP:conf/tphol/SlindN08}, and HOL-Light~\cite{DBLP:conf/tphol/Harrison09a}.
Our development was done in HOL4, inspired by existing mechanisations \cite{Katovsky,DBLP:journals/entcs/OKeefe04} in Isabelle.

Every term $t$ has a unique type $\tau$ (this is written $t:\tau$), and the most general type for an unannotated term can be inferred automatically.
A type is either a type variable or a type operator applied to argument types (an operator taking no arguments is a type constant).
Terms whose types include type variables are said to be \emph{polymorphic}.
For example, there is a polymorphic constant $=$, whose type is $\alpha\to\alpha\to\bool$, representing equality at every type $\alpha$.
In the semantics of HOL, the meaning of a type is a non-empty set, and the meaning of a (non-polymorphic) term is an element of its type.
(A polymorphic term is interpreted as a function taking a set for each type variable and returning an element of the resulting type.)

A \emph{predicate} is a term of type $\alpha\to\bool$ for some type $\alpha$.
The type $\alpha\to\bool$ can be abbreviated by $\alpha\;\mathsf{set}$: a predicate on $\alpha$ is the characteristic function of a subset of elements of type $\alpha$.
HOL4 includes much support for the representation of sets as predicates; we will use mathematical notation (such as $\{f\;x\mid P\;x\}$) for HOL sets.

Boolean connectives can be defined as functions in HOL, for example conjunction has type $\bool\to\bool\to\bool$.
In addition to the standard connectives and quantifiers, HOL includes a \emph{Hilbert choice} term that, given a predicate $P$ as argument, denotes an arbitrary (but fixed) element $c$ such that $P\; c$ holds, if any such element exists, otherwise denotes an arbitrary element (of the right type).
Hilbert choice is defined using a version of the axiom of choice, which is one of the axioms of HOL.
The primitive rules of inference include, for example, modus ponens, reflexivity of equality, and generalization (universal quantification) of free variables.
Typically, one also uses derived rules of inference that abbreviate applications of several primitive ones.

HOL provides principles for defining new types and new (term) constants.
The primitive types include function types ($\alpha\to\beta$ with arguments $\alpha$ and $\beta$) and Booleans ($\bool$).
New types can be defined (from theorems asserting predicates on existing types are non-empty).
Standard types defined in HOL include the type $\mathsf{unit}$ with a single element $()$, the option type with constructors $\mathtt{NONE}:\alpha\;\mathsf{option}$ and $\mathtt{SOME}:\alpha\to\alpha\;\mathsf{option}$, the product type operator producing a type $\alpha\times\beta$ from arguments $\alpha$ and $\beta$, and (its dual) the sum type operator $\alpha+\beta$.
Generalising sum types, it is possible to define a new type to represent a record given the desired names and types of the record fields.
New terms can be defined in terms of existing ones, like in a functional programming language.
%}
\section{Composable Morphisms and Categories}%1575/1500%{
Category theory is an abstract theory of functions and composition.
A category is an arena in which composition takes place, and consists of \emph{morphisms}, abstract functions, \emph{objects}, the possible domains and codomains of the morphisms, and a \emph{composition law}, obeying certain conditions.
We define a type of morphisms in HOL as a polymorphic record $(\alpha,\beta,\gamma)\;\mathsf{morphism}=\langle\mathtt{dom}:\alpha,\,\mathtt{cod}:\beta,\,\mathtt{map}:\gamma\rangle$.
We use a polymorphic record for categories as well, abbreviating $(\alpha,\alpha,\beta)\;\mathsf{morphism}$ by $(\alpha,\beta)\;\mathsf{mor}$:
\begin{align*}
(\alpha,\beta)\;\mathsf{category}=\langle&\mathtt{obj}:\alpha\;\mathsf{set},\,\mathtt{mor}:(\alpha,\beta)\;\mathsf{mor}\;\mathsf{set},\,\mathtt{id}:\alpha\to\beta,\\&\mathtt{comp}:(\alpha,\beta)\;\mathsf{mor}\to(\alpha,\beta)\;\mathsf{mor}\to\beta\rangle
\end{align*}
Thus an $(\alpha,\beta)\;\mathsf{category}$ has objects of type $\alpha$ and morphisms with a map field of type $\beta$.
The $\mathtt{id}$ and $\mathtt{comp}$ fields, intended to assign an identity morphism to every object and a composite morphism to every pair of composable morphisms, do not return full morphisms, because their $\mathtt{dom}$ and $\mathtt{cod}$ fields are determined by the conditions on a category.
Rather we define a function $\mathtt{id\_in}\;c\;x=\langle\mathtt{dom}=x,\,\mathtt{cod}=x,\,\mathtt{map}=c.\mathtt{id}\;x\rangle$ and $\mathtt{compose\_in}$ similarly.

As a very simple example, we might define the discrete category (which has no morphisms except identity morphisms) on a set $s$ as follows:
\begin{align*}
\mathtt{discrete\_cat}\;s=\langle&\mathtt{obj}=s,\,\mathtt{mor}=\{\mathtt{discrete\_mor}\;x\mid x\in s\},\\&\mathtt{id}=\lambda{x}.\;(),\,\mathtt{comp}=\lambda{f,g}.\;()\rangle
\end{align*}
where $\mathtt{discrete\_mor}\;x=\langle\mathtt{dom}=x,\,\mathtt{cod}=x,\,\mathtt{map}=()\rangle$.
If $s:\alpha\;\mathsf{set}$ then $\mathtt{discrete\_cat}\;s:(\alpha,\mathsf{unit})\;\mathsf{category}$.
Every category $c$ has an opposite category $c^{\mathrm{op}}$, which provides another example definition:
\begin{align*}
\mathtt{op\_cat}\;c=\langle&\mathtt{obj}=c.\mathtt{obj},\,\mathtt{mor}=\{\mathtt{op\_mor}\;f\mid f\in c.\mathtt{mor}\},\,\mathtt{id}=c.\mathtt{id},\\&\mathtt{comp}=\lambda{f,g}.\;c.\mathtt{comp}\;(\mathtt{op\_mor}\;g)\;(\mathtt{op\_mor}\;f)\rangle
\end{align*}
where $\mathtt{op\_mor}\;f=\langle\mathtt{dom}=f.\mathtt{cod},\,\mathtt{cod}=f.\mathtt{dom},\,\mathtt{map}=f.\mathtt{map}\rangle$.

We will look at two previous formalisations of category theory in detail, one by Greg O'Keefe in 2004~\cite{DBLP:journals/entcs/OKeefe04} and another by Alex Katovsky in 2010~\cite{Katovsky}.
Both also use a polymorphic record type for categories.
However, their categories have two additional fields of type $\beta\to\alpha$ intended to assign a domain and codomain to each morphism: there is no separate type of morphisms, and maps do not have (categorical) types except in the context of a category.
The advantage of keeping morphisms independent of categories (also done by Simpson~\cite{Simpson04}) is that notions like composability and composition can be defined generically then applied when there is no containing category.

Functors and natural transformations can be defined as instances of morphisms (between categories and between functors respectively) rather than as new record types.
For example, we use $(\alpha,\beta,\gamma,\delta)\;\mathsf{functor}$ to abbreviate the type \[((\alpha,\beta)\;\mathsf{category},(\gamma,\delta)\;\mathsf{category},(\alpha,\beta)\;\mathsf{mor}\to(\gamma,\delta)\;\mathsf{mor})\;\mathsf{morphism}\] whose instances we intend to be functors from an $(\alpha,\beta)\;\mathsf{category}$ to a $(\gamma,\delta)\;\mathsf{category}$.
The $\mathtt{map}$ field of a functor gives its action on morphisms; following Katovsky we omit the action on objects from the record, since it can be defined (using Hilbert choice) in terms of the action on identity morphisms.

Instances of the $\mathsf{morphism}$ record are not necessarily attached to instances of the \emph{category} record.
It could be argued, however, that all composable morphisms occur in some category.
For example, functors are the morphisms in a category of categories.
However, this notion is problematic for several reasons, one of which is that objects in our categories must all have the same type, but we make use of functors between categories of different types.
Indeed, the separation of elements into different types in the semantics of HOL, which is a feature common to all type theories, presents difficulties whenever we want to collect elements across types into a totality.

Categories are abstract structures---many different things can play the roles of objects and morphisms---so it is natural to use a polymorphic type to model categories in HOL.
The polymorphism enables the same definition of $\mathtt{id\_in}$, for example, to be used for categories with all different types.
At the same time, polymorphism prevents us from collecting all categories in a single HOL set: in forming a term of type $(\alpha,\beta)\;\mathsf{category}\;\mathsf{set}$ we fix the types $\alpha$ and $\beta$, and therefore the set denoted cannot contain, for example, an element of type $(\alpha\times\alpha,\beta)\;\mathsf{category}$.

Coquand~\cite{DBLP:conf/lics/Coquand86} has shown that a category of categories cannot exist when working in a sufficiently strong logic, because it leads to paradox.
It is a useful idea to consider, nonetheless: for example, given two categories we can define a category, their product, that has the universal property of a categorical product to which we would like to apply general theorems about products.
The usual mathematical workaround is to stratify categories into bands of different sizes, and use the category of (relatively) small categories, which is itself not small.
This stratification can be achieved, to some extent, by \emph{universe polymorphism} in Coq (a proof assistant for dependent type theory, see~\cite{DBLP:conf/tphol/Bertot08}), as in Huet and Sa{\"i}bi's mechanisation~\cite{DBLP:conf/birthday/HuetS00}.
When we define the product of an $(\alpha,\beta)\;\mathsf{category}$ and a $(\gamma,\delta)\;\mathsf{category}$, however, it has type $(\alpha\times\gamma,\beta\times\delta)\;\mathsf{category}$.
Terms with these types could not all be objects of a $(\tau_1,\tau_2)\;\mathsf{category}$ for any (fixed) types $\tau_1$ and $\tau_2$.

Let us look now at a longer example of this problem.
In category theory, the notion of \emph{isomorphism} plays a role similar to, and often more important than, equality.
An isomorphism is a morphism with an inverse (for composition), and two objects are isomorphic if there is an isomorphism between them.
We define what it means for a pair of morphisms to be inverses:
\begin{align*}
\mathtt{iso\_pair}\;c\;f\;g=f\leadsto g\operatorname{\mathtt{-:}}c\land f\circ g\operatorname{\mathtt{-:}}c=\mathtt{id}\;g.\mathtt{dom}\operatorname{\mathtt{-:}}c\land g\circ f\operatorname{\mathtt{-:}}c=\mathtt{id}\;f.\mathtt{dom}\operatorname{\mathtt{-:}}c
\end{align*}
The notation $f\leadsto g\operatorname{\mathtt{-:}}c$ abbreviates composability within a category: $f.\mathtt{cod}=g.\mathtt{dom}\land f\in c.\mathtt{mor}\land g\in c.\mathtt{mor}$.
Similarly $f\circ g\operatorname{\mathtt{-:}}c$ abbreviates composition ($\mathtt{compose\_in}$) and $\mathtt{id}\;x\operatorname{-:}c$ an identity morphism ($\mathtt{id\_in}$).
We define $\mathtt{iso}\;c\;f=\exists{g}.\;\mathtt{iso\_pair}\;c\;f\;g$.

The morphism $f$ in the definition of $\mathtt{iso}$ has type $(\alpha,\beta)\;\mathsf{mor}$, where $\alpha$ is the type of both the domain and codomain of the morphism.
Therefore if we want to consider an isomorphism between categories (a functor that is an isomorphism), we cannot use $\mathtt{iso}$ unless the functor is between categories of the same type (e.g., we could instantiate $\alpha$ with $(\gamma,\delta)\;\mathsf{category}$ then $f$ would have type $(\gamma,\delta,\gamma,\delta)\;\mathsf{functor}$).
We define another notion, $\mathtt{cat\_iso}$, to cover functors of arbitrary types, by replacing all the ``$\operatorname{\mathtt{-:}}c$'' notions above with their instances in the (undefinable) category of categories. 

Theorems stating properties of isomorphisms (like that the notion is symmetric) need to be repeated for both $\mathtt{iso\_pair}$ and $\mathtt{cat\_iso\_pair}$. 
This situation is an example of the polymorphic record approach to categories getting in the way of expressing general categorical properties.
It is possible to define isomorphisms outside of any category in terms of arbitrary notions of composition and identity, to prove theorems about the general definition, and then to specialise it both to the case within a category and to the case where the requisite category is unavailable.
We have not taken this route, partly because it seems unwieldy: such a definition would need to take two copies each of an identity map and a composition rule to ensure a fully general type.

Having looked at some of the problems due to the type system and our use of a polymorphic record, we return to describing our development.
An element of the $\mathsf{category}$ type is not necessarily a category.
We define a predicate $\mathtt{category\_axioms}:(\alpha,\beta)\;\mathsf{category}\to\mathsf{bool}$ that checks whether a record satisfies the conditions required of a category (composition is associative, every composable pair has a composite, and so on), and include a hypothesis that the predicate is true in our theorems about categories.
We can prove, for example, $\vdash\forall{c}.\;\mathtt{category\_axioms}\;c\implies\mathtt{category\_axioms}\;(\mathtt{op\_cat}\;c)$.

As noted by Katovsky, we need to impose an additional restriction that the identity and composition fields send elements outside their domain to a specific ``undefined'' value, which we call $\mathtt{ARB}$.
For example, if $x,y\notin c.\mathtt{obj}$ then we want $c.\mathtt{id}\;x=c.\mathtt{id}\;y$ even though the value of $c.\mathtt{id}\;x$ will never feature in any proof within the category $c$.
We call this kind of restriction \emph{extensionality}.
It is thanks to extensionality that two different constructions of the same category can be proved equal in HOL.
We impose the same restriction on other structures involving partial functions, like functors and natural transformations, for example the morphism map of a functor $G$ is restricted to $G.\mathtt{dom}.\mathtt{mor}$.
A non-extensional function (one that takes values other than $\mathtt{ARB}$ outside its desired domain) can easily be made extensional by changing its value at the relevant points: we use $\mathtt{restrict}\;f\;s=\lambda{e}.\;\mathtt{if}\;e\in s\;\mathtt{then}\;f\;e\;\mathtt{else}\;\mathtt{ARB}$ to restrict $f:\alpha\to\beta$ to $s:\alpha\;\mathsf{set}$.
As we will see, it is important to respect extensionality whenever possible; thus, for example, the actual definition of $\mathtt{id\_in}$ mentioned above is wrapped by a call to $\mathtt{restrict}$ on the set $c.\mathtt{obj}$ even though $c.\mathtt{id}$ is also restricted.
We define $\mathtt{is\_category}\;c=\mathtt{category\_axioms}\;c\land\mathtt{extensional\_category}\;c$, where the second conjunct ensures $c.\mathtt{id}$ and $c.\mathtt{comp}$ are extensional.
In a dependently typed setting, both the category axioms and extensionality can be encoded directly in the definition of the $\mathsf{category}$ type (as done in~\cite{DBLP:conf/birthday/HuetS00,Sozeau,Megacz} for example).

Extensionality enables us to characterise equality between natural transformations as follows
\begin{align*}
\vdash\forall{\eta,\mu}.\;&\mathtt{is\_nt}\;\eta\land\mathtt{is\_nt}\;\mu\land(\eta.\mathtt{dom} = \mu.\mathtt{dom})\land(\eta.\mathtt{cod} = \mu.\mathtt{cod})\land{}\\
&(\forall{x}.\;x\in \eta.\mathtt{dom}.\mathtt{dom}.\mathtt{obj}\implies(\eta.\mathtt{map}\;x= \mu.\mathtt{map}\;x))\implies (\eta = \mu)
\end{align*}
(and similarly for categories and functors).
Given a functor $F$, we can define an identity natural transformation whose component at $x$ is the identity morphism for $F$'s action on $x$.
Our theorem about natural transformation equality is used to prove, for example, 
\[\vdash\forall{\eta}.\;\mathtt{is\_nt}\;\eta\implies(\eta\circ\mathtt{id\_nt}\;\eta.\mathtt{dom} = \eta)\]
because our definition of natural transformation composition (the $\circ$ operator above) produces an extensional natural transformation, so comparison with $\eta$ depends only on categorical properties.
We have defined, for any two categories $c_1$ and $c_2$, the category $[c_1\to c_2]$ of functors between them with natural transformations as morphisms, where $\mathtt{id\_nt}$ provides identity morphisms.
%mention postcomp_functor as an example of something proved using functor categories?

An alternative approach to categories gives a set of morphisms, called a \emph{hom}, for each type (pair of objects) rather than giving all the morphisms at once.
This has not been formalised in HOL, but is used by Huet and Sa{\"i}bi, for example.
We can define homs after categories: $\mathtt{hom}\;c\;x\;y=\{f\mid f.\mathtt{dom}=x\land f.\mathtt{cod}=y\land f\in c.\mathtt{mor}\}$.
The approaches are essentially equivalent, because if we defined categories in terms of homs, we could then define the collection of all morphisms in a category as $\mathtt{mor}\;c=\{f\mid\exists{x,y}.\;x\in c.\mathtt{obj}\land y\in c.\mathtt{obj}\land f\in c.\mathtt{hom}\;x\;y\}$.
%An advantage of starting with homs instead might be that we could put off mentioning the collection of all morphisms.
%We will return to this point in Section~\ref{Yoneda} when describing locally small categories.

Another choice in how to formalise categories is whether to include objects at all: we might avoid the need for an $\mathtt{obj}$ field by identifying objects with identity morphisms, as explained in pages 41--43 of \cite{DBLP:books/daglib/0023249}.
We have not seen this idea mechanised before; it would be interesting to see the ramifications of doing so on a development.
%}
\section{A Category of Classes and of Sets}%1477/1400%{
\label{Set}
\newcommand{\Set}{\ensuremath{\mathbf{Set}}}
The category $\Set$ of sets has sets as objects and functions\footnote{with specified domain and codomain, so for example the empty functions $\bot:\emptyset\to\emptyset$ and $\bot:\emptyset\to\mathbb{N}$ are distinct} as morphisms.
A natural approach to defining $\Set$ in HOL is as an instance of the $\mathsf{category}$ record where $\mathtt{obj}:\alpha\;\mathsf{set}\;\mathsf{set}$ and the $\mathtt{map}$ field of each morphism is a function $\alpha\to\alpha$.
This approach is the one taken by O'Keefe, which we follow by defining a category of ``sets within a universe'':
\begin{align*}
&\mathtt{ens\_cat}\;(u:\alpha\;\mathsf{set}\;\mathsf{set})=\\
&\quad\langle\mathtt{obj}=u,\\
&\quad\phantom{\langle}\mathtt{mor}=\{f:(\alpha\;\mathsf{set},\alpha\to\alpha)\;\mathsf{mor}\mid \forall{x}.\;x\in f.\mathtt{dom}\implies f.\mathtt{map}\;x\in f.\mathtt{cod}\},\\
&\quad\phantom{\langle}\mathtt{id}=\dots,\,\mathtt{comp}=\dots\rangle
\end{align*}
(we also ensure the morphisms are extensional).
The $\mathtt{ens\_cat}\;u$ category captures one important property of the category of sets, namely, that the morphisms are functions under composition.
It has precedent in Mac Lane's textbook on category theory~\cite{MacLaneCFTWM}, which defines $\mathbf{Ens}_u$ as above for the purpose of generalising arguments in $\Set$ to larger collections $u$ than that of all sets.
However, in our polymorphic setting, $\mathtt{ens\_cat}$ does not capture the idea that \emph{every} set should be an object, even when $u$ is a universal set (i.e., constantly true predicate), and therefore fails to have many properties the category of sets ought to have.

The problem is that $\Set$ is usually formulated in an untyped setting, and when translated to the typed setting it includes functions from sets of elements of one type to those of another.
For example, there is a function from the set of strings $\{\text{``an''},\text{``object''},\text{``in''},\text{``Set''}\}$ to the set of numbers $\{2,3,6\}$ that returns the length of a string, but in HOL, using idiomatic encodings, the former set would have type $\mathsf{string}\;\mathsf{set}$ while the latter would be a $\mathsf{num}\;\mathsf{set}$, and they cannot both be instances of $\alpha\;\mathsf{set}$ at the same time (since $\mathsf{string}$ and $\mathsf{num}$ are different types).
Clearly, this is a problem at the level of types, so remains even if we take $u$ to be the universal set of type $\alpha\;\mathsf{set}\;\mathsf{set}$.

More generally, consider the relationship in any formalisation between the collection of morphisms $c.\mathtt{mor}$ in a category and the notion of a set.
If $c.\mathtt{mor}$ is itself a set, then there is no category whose morphisms are all the functions between sets because no set can contain all functions.
So if we want $\Set$ we must arrange for $c.\mathtt{mor}$ to be a \emph{class}: some kind of collection that can be larger than a set.

We can introduce classes into HOL by means of a new type, $\mathsf{zfset}$, representing sets in an axiomatic set theory such as ZFC.
Contrary to the way types are usually introduced in HOL, that is, by definition, the type $\mathsf{zfset}$ is created as a new primitive type (extending the logic) and its properties are asserted as new axioms of HOL.
The new axioms are (higher-order versions of) the axioms of ZFC.
We also define new terms for ZFC set membership ($\mathtt{in}:\mathsf{zfset}\to\mathsf{zfset}\to\mathsf{bool}$), set union, functions as graphs, and so on.
The effect on the semantics is that types must be interpreted as classes because any existing polymorphic type might be instantiated with $\mathsf{zfset}$; the constantly true predicate of type $\mathsf{zfset}\;\mathsf{set}$ now represents the class of all ZFC sets.
This method of accessing set theory within the HOL logic was introduced by Mike Gordon \cite{DBLP:conf/tphol/Gordon96} (and independently by Bob Solovay, see~\cite{DBLP:conf/ictac/Obua06}) and is implemented in both HOL4 and Isabelle~\cite{DBLP:conf/ictac/Obua06}.
We use it, following Katovksy, to define a non-polymorphic category of sets with $\mathtt{set\_cat}.\mathtt{obj}:\mathsf{zfset}\;\mathsf{set}$ and $\mathtt{set\_cat}.\mathtt{mor}:\mathsf{zfset}\;\mathsf{set}$ also (since functions between ZFC sets can be encoded as ZFC sets).

It would be reasonable to formalise all categories non-polymorphically, using sets and classes, so our original $\mathsf{category}$ record would have an $\mathtt{obj}$ field of type $\mathtt{zfset}\;\mathtt{set}$ (although it might be better to use $\mathtt{obj}:\alpha$ and a convention to only instanstiate $\alpha$ by $\mathsf{zfset}\;\mathsf{set}$ or $\mathsf{zfset}\;\mathsf{set}\;\mathsf{set}$ and so on, to allow arbitrarily large categories).
This approach matches a mathematical view of categories based on set theory extnded with proper classes (as in~\cite{MacLaneCFTWM} for example) if we take $\mathtt{zfset}$ seriously as our representation of sets.
However, in HOL we have a competing representation of sets, namely the HOL types.
If we formalise categories purely in terms of $\mathsf{zfset}$, then our categories remain disconnected from any other theories we have formalised in HOL using types such as $\mathsf{num}$ and $\mathsf{string}$.
It is for this reason we consider the polymorphic approach to be idiomatic.

Nevertheless, we need to use the ``untyped'' $\mathtt{set\_cat}$ if we want to formalise the many properties of $\Set$ that are lost on $\mathtt{ens\_cat}$.
The reason much mathematics is encoded in set theory in the first place is that the universe of sets supports the encoding of almost all mathematical objects.
We can tap into this idea by treating a HOL set (of type $\alpha\;\mathsf{set}$ for some type $\alpha$) as a ZFC set, in particular by finding a representative element of type $\mathsf{zfset}$ to which the HOL set corresponds.
Since HOL types now represent (possibly large) classes, we can only hope to encode small HOL sets.
Following ideas from Katovksy, we define a predicate $\mathtt{is\_small}:\alpha\;\mathsf{set}\to\mathsf{bool}$ by $\mathtt{is\_small}\;s=\exists{f,z}.\;\mathtt{INJ}\;f\;s\;\{x\mid x\operatorname{\mathtt{in}} z\}$, that is, a HOL set $s$ is small if there is an injection from $s$ to the (HOL) set of elements of some ZFC set.

Working with $\mathtt{ens\_cat}$ in HOL is easier than with $\mathtt{set\_cat}$, because the morphisms are based on HOL functions directly rather than encodings of ZFC functions as ZFC sets.
We would expect $\mathtt{ens\_cat}$ to be more general than $\mathtt{set\_cat}$, since it is parameterised over an arbitrary universe of sets, and, happily, we can prove that $\mathtt{set\_cat}$ is isomorphic as a category to the $\mathtt{ens\_cat}$ of small sets of ZFC sets.
Specifically, we have proved $\vdash\mathtt{cat\_iso}\;\mathtt{set\_to\_ens}$ alongside $\vdash\mathtt{is\_functor}\;\mathtt{set\_to\_ens}$, $\vdash\mathtt{set\_to\_ens}.\mathtt{dom}=\mathtt{set\_cat}$, and $\vdash\mathtt{set\_to\_ens}.\mathtt{cod}=\{s:\mathsf{zfset}\;\mathsf{set}\mid\mathtt{is\_small}\;s\}$.
%For the proof, we use the theorem that an isomorphism of categories is the same as a bijection between the classes of objects and between each hom, and the facts that ZFC sets are determined by their members, and explicitly typed ZFC functions correspond to explicitly typed HOL functions (between ZFC sets) by their action under ZFC function application.
We define $\mathtt{ens\_to\_set}$ (using Hilbert choice) as the inverse functor.
This isomorphism provides the initial link between small HOL sets and ZFC sets, but it is useless for arbitrary small HOL sets since the $\mathtt{ens\_cat}$ involved is over a universe of small HOL sets of ZFC sets, that is, of type $\mathsf{zfset}\;\mathsf{set}$ rather than $\alpha\;\mathsf{set}$.

To view small HOL sets over an arbitrary type as being in the category of ZFC sets, we must choose a ZFC set to represent each small HOL set.
We choose representatives globally, and once-and-for-all, using Hilbert choice.
First, we prove that a class is small only if it is in bijection with the elements of a ZFC set (remember the definition of smallness only required an injection):
\begin{align*}
\vdash\forall{s}.\;\mathtt{is\_small}\;s\implies\exists{b,z}.\;&\mathtt{BIJ}\;b\;s\;\{\mathtt{SOME}\;x\mid x\operatorname{\mathtt{in}}z\}\land{}\\
&\forall{x}.\;x\notin s\implies(b\;x=\mathtt{NONE})
\end{align*}
(Here we ensure extensionality of the bijection $b$ with the $\mathsf{option}$ type ($\mathtt{SOME}$ versus $\mathtt{NONE}$) rather than with $\mathtt{ARB}$, because $\mathtt{ARB}$ may be in $z$ and we would rather be able to tell whether $x\in s$ from the value of $b\;x$.)
Given this existence theorem, it makes sense to define using Hilbert choice for every small HOL set $s:\alpha\;\mathsf{set}$ a bijection $(\mathtt{zfbij}\;s):\alpha\to\mathsf{zfset}\;\mathsf{option}$ and a class of ZFC sets $(\mathtt{zfrep}\;s):\mathsf{zfset}\;\mathsf{set}$ that is the image of $(\mathtt{zfbij}\;s)$ under $\mathtt{SOME}$.
Using this bijection and its inverse, we can also define constants $\mathtt{zfel}$ and $\mathtt{elzf}$ satisfying the following theorems
\begin{align*}
&\vdash\forall{s,z}.\;\mathtt{is\_small}\;s\implies(z\operatorname{\mathtt{in}}\mathtt{zfrep}\;s\iff\exists{x}.\; x \in s \land z = \mathtt{zfel}\;s\;x)\\
&\vdash\forall{s,x}.\;\mathtt{is\_small}\;s\land x \in s\implies\mathtt{elzf}\;s\;(\mathtt{zfel}\;s\;x) = x\\
&\vdash\forall{s,z}.\;\mathtt{is\_small}\;s\land z \operatorname{\mathtt{in}} \mathtt{zfrep}\;s\implies\mathtt{zfel}\;s\;(\mathtt{elzf}\;s\;z) = z
\end{align*}

Having fixed representative ZFC sets, we can define a functor, $\mathtt{rep\_functor}\;u$, where $u:\alpha\;\mathsf{set}\;\mathsf{set}$ satisfies $\forall{s}.\;s\in u\implies\mathtt{is\_small}\;s$, from $\mathtt{ens\_cat}\;u$ to $\mathtt{ens\_cat}\;\{(s:\mathsf{zfset}\;\mathsf{set})\mid\mathtt{is\_small}\;s\}$.
The action of $\mathtt{rep\_functor}\;u$ on morphisms is given by
\begin{align*}
f\mapsto\langle&\mathtt{dom}=\{z\mid z\operatorname{\mathtt{in}}\mathtt{zfrep}\;f.\mathtt{dom}\},\,\mathtt{cod}=\{z\mid z\operatorname{\mathtt{in}}\mathtt{zfrep}\;f.\mathtt{cod}\},\\
&\mathtt{map}=\lambda{z}.\;\mathtt{zfel}\;f.\mathtt{cod}\;(f.\mathtt{map}\;(\mathtt{elzf}\;f.\mathtt{dom}\;z)\rangle
\end{align*}
Thus, the action on objects is to send a set $s$ (in $u$) to the class of ZFC sets that are members of the chosen representation of $s$ as a ZFC set.
This kind of translation back and forth between ZFC sets, classes of their elements, and the HOL sets they might represent is more pervasive in Katovksy's formalisation of $\Set$; we try to contain it all in $\mathtt{rep\_functor}\;u$ so the rest of the development is less cluttered.

The most important property of $\mathtt{rep\_functor}\;u$ is that it is an \emph{embedding}.
In general, a functor is an embedding if it is \emph{full} and \emph{faithful}.
Both notions refer to the action on morphisms of a fixed type: full means surjective on homs and faithful means injective on homs.
An embedding is \emph{essentially injective on objects}, that is, if an embedding sends two objects in the source category to the same object in the target category, then those two objects are isomorphic in the source category.
We have proved, in HOL, that embeddings are essentially injective, and that $\mathtt{rep\_functor}\;u$ is an embedding.

We cannot prove, however, that $\mathtt{rep\_functor}\;u$ is literally injective on objects (if two objects are mapped to the same place they are equal).
To do so would be to conclude that two small HOL sets $s_1$ and $s_2$ are equal just because $\mathtt{zfrep}\;s_1=\mathtt{zfrep}\;s_2$.
But our definition of $\mathtt{zfrep}$ does not ensure injectivity: there is no reason Hilbert choice cannot pick the same ZFC set to represent two small HOL sets of the same cardinality.
Indeed by considering $s_1,s_2\in u$ where $u$ is a proper class larger than the collection of all ZFC sets, which is possible when $u:\mathsf{zfset}\;\mathsf{set}\;\mathsf{set}\;\mathsf{set}$, for example, we see that $\mathtt{zfrep}$ could not be injective.
It is unfortunate, but not fatal to our later results, that we cannot ensure literal injectivity of $\mathtt{rep\_functor}\;u$ when $u$ is small.
%}
\section{The Yoneda Embedding}%857/600%{
\label{Yoneda}
\newcommand{\op}{\ensuremath{\sp{\mathrm{op}}}}
\newcommand{\blank}{\rule[0.5ex]{0.6em}{.4pt}}
The Yoneda embedding provides a representation of a category $c$ in the category of presheaves $\widehat{c}$.
A \emph{presheaf} is a functor from ${c}\op$ to the category of sets.
Thus $\widehat{c}$ is a functor category.
The importance of the Yoneda embedding comes from the fact that $\widehat{c}$ inherits much structure from the category of sets (for example, the existence of small limits and colimits) so can be used when $c$ itself lacks this structure.
The Yoneda lemma, used to prove that the Yoneda embedding is an embedding, is an important basic milestone in any development about the category of sets, and is one of the main results in both O'Keefe's and Katovksy's developments.

The Yoneda embedding sends an object $x$ in $c$ to the \emph{contravariant hom functor} $c(\blank,x)$, which in turn sends an object $y$ in $c$ to $c(y,x)$ (that is, $\mathtt{hom}\;c\;y\;x$).
In order for $c(\blank,x)$ to be a set-valued presheaf, however, every hom $c(y,x)$ must be a set.
A hom can fail to be a set if it is too large.
A category where every hom is a set is called \emph{locally small}, and it is only for locally small categories that the Yoneda embedding can be defined.
However the proof of the Yoneda lemma never mentions smallness explicitly, and the argument still goes through when we generalise from the category of sets to an arbitrary category whose objects are appropriately set-like and include the homs of $c$.

Therefore, we prove the Yoneda lemma, and define the embedding, for presheaves whose codomain is $\mathtt{ens\_cat}\;u$.
Since the Yoneda lemma deals with hom functors, we require $u$ to contain at least all the homs of $c$.
O'Keefe, who also proves the Yoneda lemma for ``($\mathtt{ens\_cat}\;u$)-valued'' presheaves, takes $u$ to be the universal set; we instead take a universe containing only the homs of $c$ (but we also define an inclusion functor that embeds $\mathtt{ens\_cat}\;u_1$ into $\mathtt{ens\_cat}\;u_2$ whenever $u_1\subseteq u_2$).
We mostly follow Katovsky's organisation of the proof, although our proofs are slightly simpler for avoiding representations as ZFC sets.
The two main definitions are of the Yoneda functor itself, and the natural transformations between hom functors it produces:
\begin{align*}
\mathtt{Yfunctor}\;c&=\langle\mathtt{dom}=c,\,\mathtt{cod}=[c\op\to\mathtt{ens\_cat}\;(\mathtt{homs}\;c)],\,\mathtt{map}=\mathtt{YfunctorNT}\;c\rangle\\
\mathtt{YfunctorNT}\;c\;f&=\langle\mathtt{dom}=c(\blank,f.\mathtt{dom}),\,\mathtt{cod}=c(\blank,f.\mathtt{cod}),\,\mathtt{map}=\lambda{x}.\;c(x,f)\rangle
\end{align*}
(Of course, we actually make restricted versions of these definitions so that the resulting functor and natural transformation are extensional.)
The notation $c(x,f)$ stands for the action of the covariant hom functor on the morphism $f$, namely \[c(x,f)=\langle\mathtt{dom}=c(x,f.\mathtt{dom}),\,\mathtt{cod}=c(x,f.\mathtt{cod}),\,\mathtt{map}=\lambda{g}.\;f\circ g\operatorname{\mathtt{-:}}c\rangle\text{.}\]
The two main theorems are that the Yoneda functor is an embedding, and is also literally injective on objects: $\vdash\mathtt{is\_category}\;c\implies\mathtt{embedding}\;(\mathtt{Yfunctor}\;c)$ and $\vdash\mathtt{is\_category}\;c\implies\mathtt{inj\_obj}\;(\mathtt{Yfunctor}\;c)$.

To get a version of $\mathtt{Yfunctor}\;c$ whose codomain is $[c^\mathrm{op}\to\mathtt{set\_cat}]$, we use $\mathtt{rep\_functor}$ and the isomorphism between $\mathtt{set\_cat}$ and the $\mathtt{ens\_cat}$ of small ZFC sets.
In the first instance, we might define
\begin{multline*}
\mathtt{zYfunctor}\;c=\\(\mathtt{postcomp\_functor}\;(c^\mathrm{op})\;(\mathtt{ens\_to\_set}\circ\mathtt{rep\_functor}\;(\mathtt{homs}\;c)))\circ\mathtt{Yfunctor}\;c
\end{multline*}
where $\circ$ is functor composition and $\mathtt{postcomp\_functor}\;b\;f$ converts a functor $f$ from $c_1$ to $c_2$ to a functor from $[b\to c_1]$ to $[b\to c_2]$.
Using our existing theorems about $\mathtt{Yfunctor}\;c$, along with general theorems about embeddings being preserved by functor composition and by lifting to a functor category, we can easily prove that $\mathtt{zYfunctor}\;c$ is an embedding whenever $c$ is locally small.
However, we cannot prove that it is injective on objects: the problem is that $\mathtt{rep\_functor}\;(\mathtt{homs}\;c)$ may not be injective on objects (as mentioned above), so if two objects in $c$ have isomorphic contravariant hom functors our embedding might represent them by the same ZFC set.

Mathematically, the Yoneda embedding ought to be (literally) injective on objects.
We can recover injectivity with a tagging trick (due to Sam Staton), by defining
\[\mathtt{zYfunctor}\;c=(\mathtt{postcomp\_functor}\;(c^\mathrm{op})\;(\mathtt{tag\_functor}\;c))\circ\mathtt{Yfunctor}\;c\]
where the action of $\mathtt{tag\_functor}\;c$ on a hom $h$ is the ZFC ordered pair whose first component is the empty ZFC set if $h$ is of the form $c(x,x)$, otherwise it is a singleton ZFC set, and whose second component is just the action of $\mathtt{ens\_to\_set}\circ\mathtt{rep\_functor}\;(\mathtt{homs}\;c)$ on $h$.
The ZFC sets now encode whether they are representing a hom of the form $c(x,x)$ or $c(x,y)$, therefore we can always tell functors $c(\blank,x)$ and $c(\blank,y)$ apart (when $x\neq y$) by their action on $x$, even if they are isomorphic functors.

We end this section with a comment about the other mechanisations' treatment of locally small categories.
O'Keefe does not extend the logic to include proper classes, so never needs to distinguish locally small categories.
Katovsky attaches, to categories intended to be locally small, an injective function that picks a ZFC set for each morphism in the category.
The set representing a hom is then given by collecting these sets-representing-morphisms up into a single ZFC set; if there is such a ZFC set, the hom is small.
This approach is problematic for at least two reasons: firstly, locally small categories are not extensional, because the same category equipped with two different representations of its morphisms would be different records in HOL; secondly, it is unnatural to invent a representation of morphisms as sets in order to prove that a category is locally small.
Indeed, there is a locally small category that cannot be represented with Katovsky's approach, namely, $\mathtt{discrete\_cat}\;\mathtt{UNIV}$ where $\mathtt{UNIV}$ is the universal set of type $(\mathsf{zfset}\to\mathsf{zfset})\;\mathsf{set}$ (the class of all functions between ZFC sets).
This category is locally small because each hom contains at most one morphism, but there is no injective representation of all functions as ZFC sets (because there are too many functions) to attach to Katovsky's record.
%}
\section{Binary Products and Other Limits}%425/250%{
\label{limits}
Like many constructions in category theory (and in abstract mathematics generally), \emph{limits} can be presented in different, equivalent ways, at different levels of generality.
One fully general definition gives a limit as a \emph{universal} \emph{cone}, and one presentation of cones give them as objects in a certain \emph{comma category}.
We have defined comma categories in HOL and used them to define cones and limits.
We have also instantiated this general definition to define binary products as limits of a functor, called $\mathtt{product\_diagram}$, from $\mathtt{discrete\_cat}\;\{1,2\}$, and then proved that these products are characterised by a less abstract and possibly more familiar description:
\begin{align*}
&\vdash\forall{c,a,b,l}.\;\mathtt{is\_category}\;c\land a\in c.\mathtt{obj}\land b \in c.\mathtt{obj}\implies\\
&\quad\mathtt{is\_limit}\;(\mathtt{product\_diagram}\;c\;a\;b)\;l\iff\\
&\quad\exists{ab,p_1,p_2}.\;(l = \mathtt{mk\_cone}\;(\mathtt{product\_diagram}\;c\;a\;b)\;ab\;(\lambda{n}.\; \mathtt{if}\;n = 1\;\mathtt{then}\;p_1\;\mathtt{else}\;p_2))\land{}\\
&\quad\quad p_1\operatorname{\mathtt{:-}}ab \to a\operatorname{\mathtt{-:}}c\land p_2\operatorname{\mathtt{:-}}ab \to b\operatorname{\mathtt{-:}}c\land{}\\
&\quad\quad\forall{ab',p_1',p_2'}.\;
p_1'\operatorname{\mathtt{:-}} ab' \to a \operatorname{\mathtt{-:}}c\land
p_2'\operatorname{\mathtt{:-}} ab' \to b \operatorname{\mathtt{-:}}c\implies\\
&\quad\quad\quad\exists!{m}.\;
m\operatorname{\mathtt{:-}}p\to ab\operatorname{\mathtt{-:}}c\land(p_1\circ m\operatorname{\mathtt{-:}}c = p_1')\land(p_2\circ m\operatorname{\mathtt{-:}}c = p_2')
\end{align*}
The notation $f\operatorname{\mathtt{:-}}x\to y\operatorname{\mathtt{-:}}c$ is shorthand for $f.\mathtt{dom}=x\land f.\mathtt{cod}=y\land f\in c.\mathtt{mor}$.

The characterisation of binary products above says that for every pair of objects $a$ and $b$ in a category $c$, there exists a canonical product object $ab$ with projections $p_1$ and $p_2$.
We can use Hilbert choice to define constants $a\times b\operatorname{\mathtt{-:}}c$ , $\pi_1^{a,b,c}$, and $\pi_2^{a,b,c}$ to denote these terms whenever they exist.
Thus we pick arbitrary, but fixed, objects and morphisms to denote ``the'' limit and its projections, which are characterised just by the universal property.
This is in line with mathematical practice.
%<readability?>

We have proved that $\mathtt{set\_cat}$ has all limits of shape $\mathtt{discrete\_cat}\;\{1,2\}$.
Ours is the first mechanisation of limits in HOL (although limits are covered by many mechanisations in other formal systems).
The fact that $\mathtt{set\_cat}$ has binary products is reassuring for the approach that models sets with a new type $\mathsf{zfset}$; we would not expect to be able to prove that $\mathtt{ens\_cat}\;u$ has limits for an arbitrary $u$ (nor for the universal set at an arbitrary type).

An important universal construction that is not an instance of a limit is the function space or exponential object.
One definition gives exponential objects via an exponential functor that is defined as \emph{right adjoint} to the binary product functor.
We can easily define the binary product functor using our work so far on binary products.
However, some of the nicer formulations of adjunctions make use of categories we cannot define.
We could always define exponential objects directly, ignoring the adjunction, but to do so is to miss out on opportunities for abstraction, for proving a theorem once and using it in many different contexts.
%}
\section{Conclusions}%361/200%{
We have looked at a new mechanisation of category theory in higher-order logic.
The foundational difficulties posed by category theory reveal themselves as the following issues during mechanisation:
\begin{itemize}
\item We cannot collect all categories as the objects of a single category.
The separation of categories by HOL types seems not to accommodate a stratification by size (one of the mathematical remedies) well.
\item
As a result, we are forced to define some things (like $\mathtt{iso}$) multiple times: once within categories, and once for every instance in higher categories that are unavailable.
\item
Since HOL functions are total but the type system is relatively weak, we need to restrict partial functions so that they are extensional, that is, comparable by their action on their domains only.
Extensionality is infectious.
\item
We cannot collect all sets up as a single set, so to formalise the category of sets we need to introduce some notion of proper classes.
\item
We need to reconcile the elements of classes, which are pure ZFC sets for us, with the existing HOL types, which represent sets at the meta-level, by choosing representative ZFC sets for small classes.
\end{itemize}

HOL is a pleasant logic in which to formalise much mathematics; it is unfortunate that it has such difficulties with highly abstract mathematics.
There has been, perhaps, more success in formalising category theory using dependent type theory.
However, the continuing experimentation with approaches to sets--some use setoids~\cite{DBLP:conf/birthday/HuetS00,Carvalho,Wilander}, others the type class mechanism in Coq~\cite{DBLP:conf/mkm/CoquandS07,Sozeau,Megacz}, and yet others import axiomatic set theory somewhat as we do~\cite{Simpson04}--suggest that dependent type theory is not necessarily the final answer (Megacz~\cite{Megacz} does not formalise large categories because ``Coq's universe polymorphism is only $\pi_1$'').

It may be better to do general abstract mathematics in a non-standard formal system.
For example, Jones suggests a non-wellfounded set theory~\cite{RBJones18}.
Gilmore and Tsinkis present category theory without size problems in an unusual natural-deduction-based set theory~\cite{DBLP:journals/tcs/GilmoreT93}. 
%<Systems and Models, Little Theories?>

The foundations of category theory involve interesting problems that have been tackled mathematically and philosophically from the beginning (e.g., \cite{Blass,springerlink:10.1007/BFb0059147}) but the answers are still not completely settled~\cite{Easwaran}.
By pursuing formalisation, we are forced to clarify the issues, and may be driven to solutions that could benefit both category theory and the practice of formalisation itself.
%}
\bibliographystyle{plain}
\bibliography{essay}
\end{document}
% vim:fdm=marker:fmr=%{,%}
