\documentclass[twoside,titlepage,11pt]{article}
\usepackage[a4paper]{geometry}
\usepackage{url,amsmath,amssymb}
\title{The Category of Sets in Higher-Order Logic}
\author{Ramana Kumar\\MPhil in Advanced Computer Science\\Peterhouse}
\begin{document}
\maketitle
\begin{abstract}%{
The foundational difficulties presented by category theory make it difficult to formalise.
Nevertheless, it is possible to formalise category theory in higher-order logic, extended with a type of small sets, in a way that fits naturally into the idiom of the logic but can still express some basic but foundationally-challenging categorical results.
It is not obvious that this approach can be pushed very far.
A more satisfying formalisation of category theory will require a new way to resolve the foundational issues, a different logic, or a new approach to formalisation.
\end{abstract}%}
\section{Introduction}%301/200%{
Category theory provides a framework for studying various notions common to diverse areas of mathematics.
It has applications in computer science ranging from the semantics of programming languages and formal systems to models of concurrency <refs>.
Since its inception, category theory has challenged the traditional foundations of mathematics, namely in set theory, by dealing with collections that are too large to be sets. 
Several workarounds have been discovered: it is often possible to set things up well for the problem at hand while failing to capture the full generality of the categorical constructions.

Formalisation is a good way to test the limits of foundational choices, because a formal development must respect those choices throughout.
There can be no resort to arguments or constructions by analogy or implied generalisation, especially in formalisations developed mechanically with a proof assistant.
In this essay, we present a new mechanisation of category theory, refining two previous developments in higher-order logic, and investigate the foundational issues.

Higher-order logic (HOL) is a simple but expressive formal system suitable for the formalisation of many pieces of mathematics.
As a result, it is the logic implemented by many proof assistants in which formal mathematics has been developed <refs>.
HOL has a standard semantics in set theory, and therefore cannot capture large categorical constructions directly.
However, it can be extended with extra axioms asserting the existence of large classes.

We formalise the category of sets, the simplest foundationally tricky category, in higher-order logic so extended.
We try to fit naturally into the idiom of the logic while still being able to express some basic but foundationally-challenging categorical results.
We comment on the limitations of our approach, compare our mechanisation to others, and make brief speculations on what a more satisfying mechanisation of category theory would require and would look like.
%}
\section{Higher-Order Logic}%460/300%{
\newcommand{\bool}{\ensuremath{\mathsf{bool}}}
Higher-order logic (HOL) is a formal system that has proved popular for formalising purely mathematical results as well as properties of computer hardware and software systems.
HOL is a natural-deduction calculus with the syntax of the simply typed lambda calculus.
A term of Boolean type is called a \emph{formula}.
HOL has four formulae specified as axioms and a dozen rules of inference; a \emph{theorem} is a formula proved from the axioms using the rules.
The HOL logic is implemented by several mechanical proof assistants, including Isabelle~\cite{DBLP:conf/tphol/WenzelPN08}, HOL4~\cite{DBLP:conf/tphol/SlindN08}, and HOL-Light~\cite{DBLP:conf/tphol/Harrison09a}.
Our development was done in HOL4, inspired by existing mechanisations \cite{Katovsky,DBLP:journals/entcs/OKeefe04} in Isabelle.

Every term has a unique type, and most general types for unannotated terms can be inferred automatically.
A type is either a type variable or a type operator applied to argument types (an operator taking no arguments is a type constant).
Terms whose types include type variables are said to be \emph{polymorphic}.
For example, there is a polymorphic constant $=$, whose type is $\alpha\to\alpha\to\bool$, representing equality at every type $\alpha$.
In the semantics of HOL, the meaning of a type is a non-empty set, and the meaning of a (non-polymorphic) term is an element of its type.
(A polymorphic term is interpreted as a function taking a set for each type variable and returning an element of the resulting type.)

A \emph{predicate} is a term of type $\alpha\to\bool$ for some type $\alpha$.
The type $\alpha\to\bool$ is sometimes abbreviated by $\alpha\;\mathsf{set}$.
This reinforces the idea that types represent sets: a predicate on $\alpha$ is the characteristic function of a set of elements of type $\alpha$.

Boolean connectives can be defined as functions in HOL, for example conjunction has type $\bool\to\bool\to\bool$.
In addition to the standard connectives and quantifiers, HOL includes a \emph{Hilbert choice} term that, given a predicate $P$ as argument, denotes an arbitrary (but fixed) element $c$ such that $P\; c$ holds, if any such element exists, otherwise denotes an arbitrary element (of the right type).
Hilbert choice is defined using a version of the axiom of choice, which is one of the axioms of HOL.
The primitive rules of inference include, for example, modus ponens, reflexivity of equality, and generalization of free variables.
Typically, one also uses derived rules of inference that abbreviate applications of several primitive ones.

HOL provides principles for defining new types and new (term) constants.
The primitive types include function types ($\alpha\to\beta$ with arguments $\alpha$ and $\beta$) and Booleans ($\bool$).
A new type can be defined after proving that a certain predicate on an existing type is non-empty.
Standard types defined in HOL include the product type operator, producing a type $\alpha\times\beta$ from arguments $\alpha$ and $\beta$, and its dual the sum type operator.
Generalising sum types, it is possible to define a new type to represent a record given the desired names and types of the record fields.
New terms can be defined in terms of existing ones, like in a functional programming language.
%}
\section{Composable Morphisms and Categories}%457/1500%{
Category theory is an abstract theory of functions and composition.
A category is an arena in which composition takes place, and consists of \emph{morphisms}, abstract functions, \emph{objects}, the possible domains and codomains of the morphisms, and a \emph{composition law}, obeying certain conditions.
We define a type of morphisms in HOL as a polymorphic record $(\alpha,\beta,\gamma)\;\mathsf{morphism}=\langle\mathrm{dom}:\alpha,\,\mathrm{cod}:\beta,\,\mathrm{map}:\gamma\rangle$.
We use a polymorphic record for categories as well, abbreviating $(\alpha,\alpha,\beta)\;\mathsf{morphism}$ by $(\alpha,\beta)\;\mathsf{mor}$:
\begin{align*}
(\alpha,\beta)\;\mathsf{category}=\langle&\mathrm{obj}:\alpha\;\mathsf{set},\,\mathrm{mor}:(\alpha,\beta)\;\mathsf{mor}\;\mathsf{set},\,\mathrm{id}:\alpha\to\beta,\\&\mathrm{comp}:(\alpha,\beta)\;\mathsf{mor}\to(\alpha,\beta)\;\mathsf{mor}\to\beta\rangle
\end{align*}
Thus an $(\alpha,\beta)\;\mathsf{category}$ has objects of type $\alpha$ and morphisms with a map field of type $\beta$.
The $\mathrm{id}$ and $\mathrm{comp}$ fields, intended to assign an identity morphism to every object and a composite morphism to every pair of composable morphisms, do not return full morphisms, because their $\mathrm{dom}$ and $\mathrm{cod}$ fields are determined by the conditions on a category, rather we define a function $\mathrm{id\_in}\;c\;x=\langle\mathrm{dom}=x,\,\mathrm{cod}=x,\,\mathrm{map}=c.\mathrm{id}\;x\rangle$ and $\mathrm{compose\_in}$ similarly.

A polymorphic record type is used in the two previous formalisations of category theory we shall consider in detail, one by Greg O'Keefe in 2004~\cite{DBLP:journals/entcs/OKeefe04} and another by Alex Katovsky in 2010~\cite{Katovsky}.
For them, the category record contains additional fields $\mathrm{dom}$ and $\mathrm{cod}$, both of type $\beta\to\alpha$, intended to assign a domain and codomain to each morphism: there is no separate type of morphisms, and maps do not have (categorical) types except in the context of a category.
<advantages of keeping morphisms separate: composability and composition can be defined generically, functors and nts are instances of morphisms>
something about defining based on homs rather than separate object and morphism sets: has not been explored in HOL, but possibly used to great effect in another system?
something about object-free categories.

<an element of category type is not necessarily a category: we impose two conditions, that it satisfies the ``axioms" predicate, and that it is \emph{extensional}, the need for which was noted by Katovsky and is...>
<in a dependently typed setting, we would not use predicates but rather encode the conditions directly in the type (as in <some related work>)>

<this polymorphic approach, using the $\mathsf{set}$ type operator for a collection, is the natural thing to do in HOL, and was done in both previous work, but is ultimately one source of foundational problems, because there is no type matching $\alpha\;\mathsf{set}$ that could represent all sets, and because the natural product of two categories would have the Cartesian product of the objects from each as its objects, but that precludes forming a category of categories with products since the objects of that would need to have both type $(\alpha,\beta)\;\mathsf{category}\;\mathsf{set}$ and $(\alpha\times\alpha,\beta)\;\mathsf{category}\;\mathsf{set}$, which is impossible, not to mention the fact that even if we didn't want it to have products, using $(\alpha,\beta)\;\mathsf{category}\;\mathsf{set}$ doesn't include categories with different types of objects in the same arena (detailed in the next section!). 
have defined a polymorphic (and hence quite weak) category of categories. the polymorphism means the same definition is used for categories of all different types, which is great, but at the same time precludes wrapping them up in a single totality, and thus we can't get products and other categorical things as mentioned above....

<more details on functors and nts. we follow O'Keefe in omitting the object function from functors, since it can be inferred. However, we did not omit the object field from the category record - nobody has mechanised a theory like that, and I'm not sure how easy it would be to make it usably familiar, but worth experimenting...
repeat need for extensionality. characterisation of functor equality. functor composition and nt composition. have defined functor categories.
%}
\section{A Category of Classes and of Sets}%239/1500%{
\newcommand{\Set}{\ensuremath{\mathbf{Set}}}
The category $\Set$ of sets has sets and objects and (typed) functions as morphisms.
A natural approach to defining $\Set$ in HOL is as an instance of the $\mathsf{category}$ record where $\mathrm{obj}:\alpha\;\mathsf{set}\;\mathsf{set}$ and the map field of each morphism is a function $\alpha\to\alpha$.
This approach is the one taken by O'Keefe.
It captures one important property of the category of sets, namely, that the morphisms are functions under composition.
However, it does not capture the idea that \emph{every} set should be an object, and therefore fails to have many properties the category of sets ought to have.
The problem is that $\Set$ is usually formulated in an untyped setting, and when translated to the typed setting it includes functions from sets of elements of one type to those of another.
For example, there is a function from the set of strings $\{\text{``an''},\text{``object''},\text{``in''},\text{``Set''}\}$ to the set of numbers $\{2,3,6\}$ that returns the length of a string, but in HOL, using idiomatic encodings, the former set would have type $\mathsf{string}\;\mathsf{set}$ while the latter would be a $\mathsf{num}\;\mathsf{set}$, and they cannot both be instances of $\alpha\;\mathsf{set}$ at the same time (since $\mathsf{string}$ and $\mathsf{num}$ are different types).
Clearly, this is a problem at the level of types, so remains even if we take $\mathrm{obj}$ to be the universal set of type $\alpha\;\mathsf{set}\;\mathsf{set}$.

The solution in Katovksy's development is to define $\Set$ in a completely non-polymorphic way, by taking $\mathrm{obj}:\mathsf{zfset}\;\mathsf{set}$ where $\mathsf{zfset}$ is a new type representing sets in the axiomatic set theory ZFC <ref>.
Contrary to the way types are usually introduced in HOL, that is, by definition, the type $\mathsf{zfset}$ is created as a new primitive type (extending the logic) and its properties are asserted as new axioms of HOL.
The new axioms are (higher-order versions of) the axioms of ZFC.
The effect on the semantics is that types must be interpreted as classes (that is, possibly larger than sets), and the universe of HOL types becomes correspondingly larger, since any existing polymorphic type might be instantiated with $\mathsf{zfset}$.
This method of accessing set theory directly within the HOL logic was introduced in <Mike Gordon HOL-ST> and is implemented in both HOL4 and Isabelle/HOL <Obua HOLZF>.
%}
\section{Binary Products and Other Limits}%5/250%{
%}
\section{The Yoneda Embedding}%3/500%{
%}
\section{Conclusions}%18/500%{
<summary of issues facing mechanisations>
<related work in other systems esp. that can solve major foundational problems>
%}
\bibliographystyle{plain}
\bibliography{essay}
\end{document}
% vim:fdm=marker:fmr=%{,%}
